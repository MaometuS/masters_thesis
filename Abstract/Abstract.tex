 %% Abstract.tex

%%%%%%%%%%%%%%% Abstract %%%%%%%%%%%%%%%

\cleardoublepage
\begin{abstract}
%\renewcommand{\baselinestretch}{1.2}\normalsize
\renewcommand{\baselinestretch}{1.2}\abstractfont

Computer vision models have been instrumental in detection of anomalies in industrial manufacturing processes. Automation of the detection of anomalies plays a crucial role in reducing the production cost while maintaining quality control and ensuring the consistency and reliability of production lines. Current models are capable of categorizing the products as anomalous or nominal, moreover the models can outline the defective region with high accuracy.

% anomaly detection

Contemporary industrial anomaly detection models utilize multiple types of approaches. Three successful methods are: k-nearest neighbor method where statistical outliers are detected on the feature space of a pre-trained network, reconstruction based methods where a decoder is trained to predict the input of an encoder in the form of pre-trained network, student-teacher base methods where a student network is trained to match the output of the teacher pre-trained network.

% problem
As was mentioned, the use of pre-trained networks is prevalent in industrial anomaly detection models. The lack of large industrial datasets and the scarcity of samples with anomalies in those datasets make the use of classical learning methods impractical. Therefore, in the recent year, there has been an emergence of industrial anomaly detection models that rely on the features extracted by pre-trained neural networks. Out of all available pre-trained models for use as a backbone in industrial anomaly detection models, models trained on the ImageNet1k classification task are the dominant choice. ImageNet1k is a general-purpose dataset that consists of 1,000 different classes of 1,281,167 training images. While the usage of ImageNet1k-trained backbones is the dominant idea and has been shown to be effective in most cases, this thesis proposes that training on ImageNet1k generates feature spaces that are general purpose, and better performance of industrial anomaly detection models could be achieved with more specialized feature spaces generated by various methods. This investigation aims to explore and validate these alternative feature spaces to enhance model performance in detecting industrial anomalies.

% method
In this thesis we explore methods of forming specialized feature spaces for industrial anomaly models with the purpose of improving their performance. The main proposal of this thesis is to investigate if a dataset can be formed such that, features extracted from such a dataset would show an improvement in the accuracy of anomaly detection models. The proposed method is to extract a subset of images from a large image dataset such as Laion4b, Laion400m, YFCC100M etc. while ensuring the relevancy of the images that are being extracted to the industrial use cases. To achieve that task, we train an "extractor" bi-class(which are industrial and non-industrial) classifier model, and apply the extractor model to the large image dataset. To train the "extractor" model we form another dataset that contains images consisting of industrial and non-industrial classes.

% check other backbones
In addition to the generation of an industrial-specialized dataset, this thesis explores the effect of other feature spaces that have not been widely utilized in industrial anomaly detection models. Specifically, during our experiments, it was discovered that Vision Transformer-based models as backbones tend to perform considerably worse compared to CNN-based models. This observation highlights the limitations of Vision Transformer models in capturing the intricate patterns and features necessary for effective anomaly detection in industrial settings. Furthermore, out of all available CNN-based models, ResNet-based models showed superior performance, demonstrating their robustness and effectiveness in this domain. More precisely, WideResNet50 and WideResNet101 models proved to achieve the highest accuracy when used as backbones for industrial anomaly detection models.

% Evaluation
To evaluate the performance of different pre-trained models as backbones, we test them using with different types of industrial anomaly detection models. Throughout all the experiments we use MVTechAD dataset, which contains over 5000 images of industrial objects consisting of 15 different categories. For the experiments of this thesis we use models RealNet and PatchCore to represent a model per each type of industrial anomaly detection models. RealNet for reconstruction based methods and PatchCore for feature-embedding based methods.

%First exp
To train the "extractor" models we set up a dataset consisting of 2 classes, industrial and non-industrial. Main part of the dataset consist of ImageNet1k classes manually separated into two categories. We train our extractor ResNet50 on the constructed dataset with the classification task. The resulting extractor model then used to extract industrial class images from a large image dataset, in our case YFCC100M. This results in a unlabeled dataset consisting of more that 6 million images. Due to the unlabeled nature of the generated dataset, we use self-supervised learning methods to achieve feature extraction.  

%Second exp
Although it is common for Vision Transformer models to be used self-supervised learning, experiments showed that, using ViT models as backbones are ineffective compared to CNN models. Therefore, we train WideResNet50 using DINO self-supervised learning pipeline on the dataset formed from applying the extractor model on the YFCC100M dataset. We perform self-supervised learning with the same conditions on the randomly extracted dataset. Three WideResNet50 models are compared using as backbones to two industrial anomaly detection models, namely PatchCore and RealNet.

% Conclusion
In conclusion, this thesis demonstrates that the choice of feature space and backbone architecture significantly impacts the performance of industrial anomaly detection models. By generating a specialized dataset and exploring alternative feature extraction methods, we have shown that more tailored feature spaces can enhance detection accuracy. Our experiments confirmed the superiority of CNN-based models, particularly WideResNet50 and WideResNet101, over Vision Transformer models. The findings highlight the potential of self-supervised learning on industrial-specific datasets to further improve anomaly detection capabilities. Future research should continue to refine these approaches, potentially leading to more robust and efficient industrial anomaly detection systems.

%---Put blank line before the "end" command-----%

\end{abstract}